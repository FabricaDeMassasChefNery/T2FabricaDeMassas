
\chapter[Contexto do Negócio]{Contexto Negócio \\ (Chef Nery)}
A  fábrica de massas ChefNery surgiu em 2015 a partir de um hobbie de Pedro Nery um apaixonado por massas.  A empresa atualmente, apresenta uma grande variedade de produtos e sendo alguns exclusivos. Com o crescimento grande acentuado a empresa começou a atender parlamentares da câmara e senado federal.\\
\tab O objetivo central deste projeto é permitir por meio de ferramentas tecnológicas recursos que possibilitem a continuidade do crescimento da fábrica de massas. Nossa proposta deve englobar tópicos como controle de estoque, armazenamento de pedidos, agendamento de pedidos, e-commerce, e uma integração com outras ferramentas de comunicação como o Facebook.\\
\tab Este projeto era um dos projetos de expansão de Pedro Nery (dono da fábrica de massas) tanto que o domínio chefnery.com foi adquirido anos atrás.\\
\tab Neste projeto, a partir do processo elaborado e utilizando várias atividades, pretende-se elaborar os requisitos de forma que melhor atenda ao cliente nesse contexto, e assim elaborar um produto que seja o esperado pelo cliente. \\
\tab Na sessão subsequente será apresentado o problema enfrentado pelo cliente e a proposta de solução incluída neste trabalho. Ressalta-se a existência do apêndice A, o qual, encontra-se disponível o documento de visão sobre este projeto. \\


{\large {\section { Problema \\ } } }

Atualmente, existe uma série de ferramentas, tecnologias e metodologias que auxiliam no processo de gestão de negócios.  Assim, o uso destas técnicas podem impactar positivamente na gestão da empresa. Com o objetivo de identificar  pontos a serem melhorados ou dificuldades atualmente enfrentadas pela empresa foi proposto um diagrama de Fishbone elaborado pela Equipe de Requisitos e o cliente. A seguir, a imagem apresenta o diagrama descrito anteriormente.\\
\tab Como consequência do diagrama de Fishbone, foi obtido pontos a serem otimizados no funcionamento da fábrica de massas, tais como:
\begin{enumerate}
	\item Controle dos pedidos;
	\item Controle de estoque;
	\item Ampliação para e-commerce;
	\item Ferramenta para comunicação com o cliente;
\end{enumerate}


{\large {\section { Solução \\ } } }

A solução para este problema foi elaborado junto ao cliente. No caso, foi proposto um site responsivo, ou seja, que se adapta aos diferentes tipos de dispositivos (normalmente, relacionado com a interface gráfica).  Além disso, acordou-se com o cliente uma série de características que este site deveria ter:

\begin{enumerate}
	\item Ferramenta para agendamento de Pedidos: Permitir que o usuário faça pedidos para um determinado evento;
	\item Ferramenta de controle de estoque: Ferramenta para o controle do estoque;
	\item Possibilidade de realizar pedidos via web: Realizar pedidos no mesmo dia e recebê-los para uma determinada reunião;
	\item Apresentar produtos e suas características;
	\item Web-commerce: Possibilitar a venda dos produtos  via internet;
\end{enumerate}

\tab Na seção posterior estão apresentados os requisitos de forma mais detalhada e precisa.

