
\chapter[Introdução]{Introdução}

A definição precisa dos requisitos é fundamental  no desenvolvimento de sofware, nesto contexto Lopes e  Audy, destacam:

\begin{quotation}
\textit{O sucesso no desenvolvimento de um software é medido principalmente pela forma com que ele realiza a tarefa para qual foi proposto. O esforço de desenvolvimento é total ou parcialmente desperdiçado se o software, por melhor que seja a qualidade de sua codificação, não cumpre com a tarefa que foi destinado.}
\end{quotation}

Assim, este trabalho tem como finalidade desenvolver os requisitos da fábrica de massa ChefNery. Para isso, utilizou-se o SAFe como referência, o processo utilizado se encontra disponível como anexo \ref{Processo}.  Além disso, para controlar o prazo de realização das atividades foi definido um cronograma que se encontra disponível com anexo \ref{Visao}. Outros artefatos providos do processo, tais como a Especificação Suplementar \ref{Suplementar} estão disponíveis como anexos. \\
\tab Para realizar as atividades de gerenciamento de requisitos e outras atividades foram utilizados as ferramentas: RallyDev \\
\tab  Este trabalho encontra-se dividido em 6 seções: Introdução; Contexto do negócio; Gerenciamento de Requisitos; Sprints; Conclusão; e Apêndices.\\