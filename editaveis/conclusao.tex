\chapter[Conclusão]{Conclusão}

\section{Sobre a execução do projeto}
Concluir este projeto foi, seguramente, um grande desafio para o time. Grande parte do conhecimento utilizado fora adquirido durante o tempo disponível para desenvolver os requisitos e muita coisa foi aperfeiçoada neste mesmo período. Tendo esta consideração, pode-se dizer que este primeiro projeto de gerência de requisitos teve seus altos e baixos até alcançar sua forma final, que pode não estar perfeita, mas a experiência adquirida é válida e os membros do time aprenderam bastante.

\section{Sobre a interação com o cliente}
	A experiência mais inédita neste projeto certamente foi trabalhar diretamente com um cliente não relacionado ao meio acadêmico (ao qual o time está habituado), como geralmente ocorre em projetos reais.
Houve interação regularmente com o cliente durante o período de planejamento e desenvolvimento do projeto e ficaram claros os desafios que a fase de elicitação de requisitos apresenta; em especial, a dificuldade em alcançar entendimento e concordância plena entre as partes envolvidas na elicitação. Foi preciso muito esclarecimento e às vezes repetição das mesma idéias até que surgisse um consenso geral.
Naturalmente, não utilizamos todas as técnicas de elicitação. Neste sentido, pode-se dizer que falta experiência prática do time em relação a outras técnicas que não foram utilizadas.

\section{Sobre a disciplina}
A disciplina de Requisitos de Software, principal motivo da elaboração do projeto, forneceu uma parte considerável do conhecimento adquirido ao longo de cada etapa percorrida. Durante o semestre inteiro, o time foi orientado pela teoria transmitida em sala de aula e pelos monitores, que estiveram disponíveis para tirar dúvidas técnicas e teóricas.
O esquema de pontos de controle foi extremamente útil para cadenciar o ritmo de trabalho da equipe e fornecer \textit{feedback} corretivo sobre o trabalho feito até então.
A forma com que a disciplina tirou os membros do time de sua zona de conforto, para trabalhar com clientes reais (embora não num projeto comercial real) propiciou grande aprendizado sobre Engenharia de software no que tange as relações sociais com colegas e clientes, além, claro, da gerência de requisitos.
