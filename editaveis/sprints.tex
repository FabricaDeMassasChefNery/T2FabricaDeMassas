
\chapter[Sprints]{Sprints}


\section{\large{Sprint 1}}

\subsection{Planejamento}
    O objetivo da \textbf{Sprint} 1 foi detalhar, validar e priorizar os \textbf{temas de investimento, épicos, features e histórias de usuário}.

\subsection{Resumo}
    Numa reunião o cliente, foram detalhados e priorizados os temas de investimento, épicos e features. Com estes dados, foram geradas histórias de usuário correspondentes ao que o time entendeu ser o sistema. Estas histórias de usuário foram então validadas e priorizadas numa segunda reunião.

    As seguintes \textbf{features} foram pririzadas:

    % \begin{table}[]
    % \centering
    % \caption{My caption}
    % \label{my-label}
    % \begin{tabular}{l}
    % \textbf{Identificação:} (E1F3) – Gestão de Produtos                                                                                                                                                                                  \\
    % \textbf{Descrição:} Gerir os produtos da Fábrica de Massas do Chef Nery com integração ao banco de dados do sistema. Deve informar o valor de cada produto e gerar informações relevantes de acordo com o tempo acerca dos produtos.
    % \end{tabular}
    % \end{table}
    %
    % \begin{table}[]
    % \centering
    % \caption{My caption}
    % \label{my-label}
    % \begin{tabular}{l}
    % \textbf{Identificação:} (E2F4) – Divulgar Canais de Comunicação                                              \\
    % \textbf{Descrição:} Disponibilizar informações do whatsapp e facebook da empresa a partir do site da empresa
    % \end{tabular}
    % \end{table}
    %
    % As seguintes \textbf{histórias de usuário} foram priorizadas:
    %
    % \begin{table}[]
    % \centering
    % \caption{My caption}
    % \label{my-label}
    % \begin{tabular}{l}
    % \textbf{Identificação:} F1H3 - Efetuar pedido checando o estoque                                                                                                                  \\
    % \textbf{Descrição:} Eu como administrador do estoque desejo que o próprio sistema avalie a ausência de produtos, a fim de permitir apenas pedidos com a possibilidade de entrega.
    % \end{tabular}
    % \end{table}
    %
    % \begin{table}[]
    % \centering
    % \caption{My caption}
    % \label{my-label}
    % \begin{tabular}{l}
    % \textbf{Identificação:} F1H5 - Ser alertado quanto a pedido                                                                                                         \\
    % \textbf{Descrição:} Eu como cliente e como vendedor desejo ser alertado quando o pedido foi efetuado para que tenha clareza quanto o status de aceitação do pedido.
    % \end{tabular}
    % \end{table}
    %
    % \begin{table}[]
    % \centering
    % \caption{My caption}
    % \label{my-label}
    % \begin{tabular}{l}
    % \textbf{Identificação:} F3H1 - Manter Produto                                                                                                                                                                                                                                                                                                                                                                                              \\
    % \textbf{Descrição:} Como vendedor responsável pela Fábrica de Massas eu quero criar, atualizar, ler e deletar produtos para que seja possível ter um controle sobre quais e quantos produtos estão disponíveis no meu site.Será necessário identificar cada produto com as seguintes informações:
    % \begin{itemize}
    %     \item Nome do Produto;
    %     \item Ingredientes;
    %     \item Categoria (Culinária Italianas, Culinária Oriental, Culinária Árabe);
    %     \item Preço do Produto;
    %     \item Quantidade Disponível.
    % \end{itemize}
    % \end{tabular}
    % \end{table}
    %
    % \begin{table}[]
    % \centering
    % \caption{My caption}
    % \label{my-label}
    % \begin{tabular}{l}
    % \textfbf{Identificação:} F3H2 - Calcular Valor do Produto                                                                                                                                                                                                                                                                                                                                                                             \\
    % \textbf{Descrição:} Como vendedor do produto desejo saber quanto custa cada produto baseado nos ingredientes e suas quantidades para que seja possível verificar o valor do produto. Esse custo será chamado de custo operacional.,Opcionalmente, gostaria de saber o valor do produto baseado também no custo da gasolina do transporte para a entrega e também o próprio custo de tempo gasto pelo Chef Nery na produção do mesmo.
    % \end{tabular}
    % \end{table}
    %
    % ///
    %


    \begin{table}[H]
    \centering
    \caption{Sprints}
    \begin{tabular}{c|p{10cm}}
    \hline
    \textbf{Identificação:} (E1F3) – Gestão de Produtos \\
    \textbf{Descrição:} Gerir os produtos da Fábrica de Massas do Chef Nery com integração ao banco de dados do sistema. Deve informar o valor de cada produto e gerar informações relevantes de acordo com o tempo acerca dos produtos.    \\
    \end{tabular}
    \end{table}
