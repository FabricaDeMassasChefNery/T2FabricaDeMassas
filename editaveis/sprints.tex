
\chapter[Sprints]{Sprints}


\section{\large{Sprint 1}}

\subsection{Planejamento}
    O objetivo da \textbf{Sprint} 1 foi detalhar, validar e priorizar os \textbf{temas de investimento, épicos, features e histórias de usuário}.

\subsection{Resumo}
    Numa reunião o cliente, foram detalhados e priorizados os temas de investimento, épicos e features. Com estes dados, foram geradas histórias de usuário correspondentes ao que o time entendeu ser o sistema. Estas histórias de usuário foram então validadas e priorizadas numa segunda reunião.

    As seguintes \textbf{features} (identificadas por E\textbf{n}F\textbf{n}) e \textbf{histórias de usuário} (identificadas por F\textbf{n}H\textbf{n}) foram priorizadas:

    \begin{table}[H]
    \centering
    \begin{tabular}{c|p{10cm}}
    \hline
    \textbf{Identificação:} E1F3 – Gestão de Produtos \\
    \hline
    \textbf{Descrição:} Gerir os produtos da Fábrica de Massas do Chef Nery com integração ao banco\\
     de dados do sistema. Deve informar o valor de cada produto e gerar informações relevantes de \\
     acordo com o tempo acerca dos produtos.    \\
    \hline
    \end{tabular}
    \end{table}

    \begin{table}[H]
    \centering
    \begin{tabular}{c|p{10cm}}
    \hline
    \textbf{Identificação:} E2F4 – Divulgar Canais de Comunicação \\
    \hline
    \textbf{Descrição:} Disponibilizar informações do whatsapp e facebook da empresa a partir do \\
    site da empresa.\\
    \hline
    \end{tabular}
    \end{table}

    \begin{table}[H]
    \centering
    \begin{tabular}{c|p{10cm}}
    \hline
    \textbf{Identificação:} F1H3 - Efetuar pedido checando o estoque \\
    \hline
    \textbf{Descrição:} Eu como administrador do estoque desejo que o próprio sistema avalie a \\
    ausência de produtos, a fim de permitir apenas pedidos com a possibilidade de entrega. \\
    \hline
    \end{tabular}
    \end{table}

    \begin{table}[H]
    \centering
    \begin{tabular}{c|p{10cm}}
    \hline
    \textbf{Identificação:} F1H5 - Ser alertado quanto a pedido \\
    \hline
    \textbf{Descrição:} Eu como cliente e como vendedor desejo ser alertado quando o pedido foi \\
    efetuado para que tenha clareza quanto o status de aceitação do pedido.\\
    \hline
    \end{tabular}
\end{table}

\begin{table}[H]
    \centering
    \begin{tabular}{c|p{10cm}}
    \hline
    \textbf{Identificação:} F3H1 - Manter Produto \\
    \hline
    \textbf{Descrição:} Como vendedor responsável pela Fábrica de Massas eu quero criar, atualizar,\\
     ler e deletar produtos para que seja possível ter um controle sobre quais e quantos produtos \\
      estão disponíveis no meu site. \\
    Será necessário identificar cada produto com as seguintes informações: Nome do Produto, \\
    Ingredientes, Categoria (Culinária Italianas, Culinária Oriental, Culinária Árabe), \\
     Preço do Produto e Quantidade Disponível;\\
    \hline
        \end{tabular}
\end{table}

    \begin{table}[H]
    \centering
    \begin{tabular}{c|p{10cm}}
    \hline
    \textbf{Identificação:} F3H2 - Calcular Valor do Produto    \\
    \hline
    \textbf{Descrição:} Como vendedor do produto desejo saber quanto custa cada produto baseado nos\\
     ingredientes e suas quantidades para que seja possível verificar o valor do produto. Esse custo\\
      será chamado de custo operacional.\\
	Opcionalmente, gostaria de saber o valor do produto baseado também no custo da gasolina do\\
     transporte para a entrega e também o próprio custo de tempo gasto pelo Chef Nery na produção\\
      do mesmo.  \\
    \end{tabular}
    \end{table}

    \begin{table}[H]
    \centering
    \begin{tabular}{c|p{10cm}}
    \hline
    \textbf{Identificação:} F3H3 - Gerar Relatório de Caixa \\
    \hline
    \textbf{Descrição:} Como administrador da Fábrica de Massas quero saber a partir de um gráfico\\
     informações de acordo com o tempo acerca do fluxo de caixa de minha empresa envolvendo produtos\\
      vendidos para que seja possível ter um melhor controle da minha empresa. Devem ser\\
       disponibilizadas opções de visualização diária, semanal e anual.\\
Opcionalmente com projeções futuras dos fluxos de caixa e a possibilidade de comparar fluxos de caixa.\\
    \hline
    \end{tabular}
    \end{table}

    \begin{table}[H]
    \centering
    \begin{tabular}{c|p{10cm}}
    \hline
    \textbf{Identificação:} Identificação: F4H1 - Apresentar informações sobre a empresa e contatos \\
    \hline
    \textbf{Descrição:} Como vendedor responsável pela Fábrica de Massas eu quero disponibilizar\\
     informações para contato e atalhos para outros meios comunicação como:
      facebook; \\
      Enviar e-mail direto ao usuário responsável;\\
      Apresentar atalho para a página oficial da empresa no Facebook;\\
      Apresentar informações de contato como email e telefone.\\
      \hline
    \end{tabular}
    \end{table}
