
\chapter[Sprints]{Sprints}


\section{\large{Sprint 1}}

\subsection{Planejamento}
    O objetivo da \textbf{Sprint} 1 foi detalhar, validar e priorizar os \textbf{temas de investimento, épicos, features e histórias de usuário}.

\subsection{Resumo}
    Numa reunião o cliente, foram detalhados e priorizados os temas de investimento, épicos e features. Com estes dados, foram geradas histórias de usuário correspondentes ao que o time entendeu ser o sistema. Estas histórias de usuário foram então validadas e priorizadas numa segunda reunião.

    As seguintes \textbf{features} foram pririzadas:

    \begin{table}[]
    \centering
    \caption{My caption}
    \label{my-label}
    \begin{tabular}{l}
    \textbf{Identificação:} (E1F3) – Gestão de Produtos                                                                                                                                                                                  \\
    \textbf{Descrição:} Gerir os produtos da Fábrica de Massas do Chef Nery com integração ao banco de dados do sistema. Deve informar o valor de cada produto e gerar informações relevantes de acordo com o tempo acerca dos produtos.
    \end{tabular}
    \end{table}
