
\chapter[Referências]{Referências}

Structured programming  Academic Press Ltd. London, UK, UK ©1972\\

INSTITUTE OF ELECTRICAL AND ELETRONICS ENGINEERS. Guide to the Software Engineering Body of Knowledge, Version 3. [S.l.], 2014. 34-38 p.\\

BECK, K. et al. Agile manifest. 2001. Disponível em: \\<http://www.agilemanifesto.org>.\\

BOEHM, B.; TURNER, R. Balancing Agility and Discipline: A Guide for the Perplexed.\\

2001-2012 Scott W. Ambler \\<http://www.agilemodeling.com/essays/agileDocumentation.htm>\\

Martins L. E. G. e Daltrini B. M., Utilização dos preceitos da Teoria da Atividade na Elicitação dos Requisitos de Software, SBES’1999, pp.\\

FORTEZZA Consulting. 2014. Disponível em: <http://www.fortezzaconsulting.com/blog/when-to-use-agile/>

<https://www.rallydev.com/>, acesso em 26 de Setembro de 2016.\\
Boehm, B. Software Engineering Economics, Prentice-Hall, 1981.\\

Goguen, J. A. and C. Linde “Techniques for Requirements Elicitation”, Software Requirements Engineering, 2nd. Ed., IEEE CS Press, 1997, pp 110-122.\\

Lopes,  Leandro;  Audy,  Jorge  Luis  Nicolas.  Em  busca  de  um  modelo  de  referencia  para 
engenharia  de  requisitos  em  ambiente  de  desenvovimento  distribuido  de  software.  In: 
WER'03 - Workshop em Engenharia de Requisitos. Piracicaba, SP. RJ:\\

Portella Cristiano R.R; Técnicas de prototipação na especificação de requisitos e sua influência na qualidade do software. Dissertação de Mestrado, Instituto de Informática PUC-Campinas, Campinas, 1994.\\

THAYER, R.H. e DORFMAN, M.; “Introduction to Tutorial Requirements Engineering” in Software Requirements Engineering. IEEE-CS Press, Second Edition, 1997, p.p. 1-2.\\

BELGAMO, Anderson; MARTINS, Luiz Eduardo Galvão. Estudo Comparativo sobre as técnicas de Elicitação de Requisitos do Software. In: XX Congresso Brasileiro da Sociedade Brasileira de Computação (SBC), Curitiba–Paraná. 2000.\\

DE GRANDE, José Inácio; MARTINS, Luiz Eduardo Galvão. SIGERAR: Uma Ferramenta para Gerenciamento de Requisitos. In: WER. 2006. p. 75-83.\\

GENVIGIR, Elias Canhadas. Um modelo para rastreabilidade de requisitos de software baseado em generalização de elos e atributos. São José dos Campos: Instituto Nacional de Pesquisas Espaciais, 2009.\\

Gotel, O. and Finkelstein, A. (1997) “Extended requirements traceability: Results of an industrial case study”. In Proceedings of the 3rd IEEE International Symposium on Requirements Engineering, Washington, DC, p. 169.\\
