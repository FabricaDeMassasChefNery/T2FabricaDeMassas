\begin{apendicesenv}

\partapendices

\chapter{Documento de Visão}

{\large {\section { Introdução \\ } } }

{\subsection {Finalidade\\ }}
\tab Este documento contém os fatores mais relevantes que levarão à construção do site Chefe Nery. Esse projeto será desenvolvido pelos estudantes das disciplinas requisitos de software que é ministrada pela professora Elaine Venson da Universidade de Brasília, campus Gama/DF (FGA). Serão apresentadas aqui todas as principais características, finalidades e motivações para o desenvolvimento dessa aplicação, assim como as limitações e empecilhos que existem já na concepção da ideia.
\\

{\subsection {O problema\\ }}
\textbf{Discussão:} A fábrica de massas do Chefe Nery é uma micro-empresa que fornece produtos alimentícios em setores como culinária italiana, culinária oriental, entre outros. Realiza pedidos principalmente na Câmara dos Deputados, local de trabalho do dono da empresa e outras regiões próximas ao Plano Piloto e Park Way. Os pedidos são feitos majoritariamente pela plataforma mobile Whatsapp e com registros de pedidos não muito eficiente. \\
\textbf{Formulação do Problema:} 


\newcommand{\nextitem}{\par\hspace*{\labelsep}\textbullet\hspace*{\labelsep}}

\begin{tabular}{|l|p{3in}|}
  \hline
  \textbf{O problema:} & \nextitem Pedidos feitos manualmente
    \nextitem Falta de divulgação da empresa 
    \\ \hline
  \textbf{Afeta:} & \nextitem A gestão de pedidos 
    \nextitem Estoque da empresa 
    \\ \hline
  \textbf{Cujo impacto é:} & \nextitem Possível perda de pedidos
    \\ \hline
  \textbf{Uma boa solução seria:} & \nextitem Um sistema de plataforma web que realize os pedidos de forma autônoma e informasse à empresa sobre as entregas a serem feitas, além de informar aos clientes sobre os produtos preparados pelo chefe Nery.
    \\ \hline

\end{tabular}

{\large {\section { Abordagem do problema \\ } } }

\textbf{Discussão:} A fábrica de massas do Chefe Nery é uma micro-empresa que fornece produtos alimentícios em setores como culinária italiana, culinária oriental, entre outros. Realiza pedidos principalmente na Câmara dos Deputados, local de trabalho do dono da empresa e outras regiões próximas ao Plano Piloto e Park Way. Os pedidos são feitos majoritariamente pela plataforma mobile Whatsapp e com registros de pedidos não muito eficiente. \\

\textbf{Sentença de Posição do Produto:}  \\

\begin{tabular}{|l|p{3in}|}
  \hline
  \textbf{Para} & \nextitem Clientes que passam agora a pedir pelo sistema.
    \nextitem Administradores do estoque, uma vez o próprio sistema permite entregas caso haja estoque. 
    \\ \hline
  \textbf{Que} & \nextitem Precisam realizar pedidos via internet 
    \\ \hline
  \textbf{O produto} & \nextitem A aplicação do Chefe Nery
    \\ \hline
  \textbf{Faz} & \nextitem Expõe os produtos
  \nextitem Gerencia os pedidos
  \nextitem Cadastro de clientes
  \nextitem Realiza pagamentos (caso solicitado)
    \\ \hline

\end{tabular}

{\large {\section { Partes Envolvidas \\ } } }
\textbf{Resumo dos Stakeholders:}  \\

\begin{tabular}{|p{2in}|p{2in}|p{2in}|}
  \hline
  \textbf{Nome} & \textbf{Descrição} & \textbf{Responsabilidades} \\ \hline
  Equipe de Desenvolvimento & Fazem parte do time de desenvolvedores & Responsáveis pela elaboração do programa \\ \hline
  Administradores & Quem administra o estoque e/ou pedidos & Responsável pela gerência de pedidos e estoques da empresa \\ \hline

\end{tabular}

\textbf{Resumo dos Usuários:}  \\

\begin{tabular}{|l|p{3in}|}
  \hline
  \textbf{Nome} & \textbf{Descrição} \\ \hline
  Clientes da fábrica de massas & Os clientes que irão fazer os pedidos pelo site \\ \hline
  Administradores & Os administradores da plataforma \\ \hline

\end{tabular}

\textbf{Principais necessidades dos Usuários:} Os usuários necessitam de um canal que os possibilitem pesquisar os produtos existentes no site, escolher os produtos, assim como sua quantidade, o local de entrega dos produtos e identificar a forma de pagamento que julgarem necessária.   \\


\chapter{Especificação Suplementar}

{\large {\section { Introdução \\ } } }

Apesar do processos de desenvolvimento ter sido baseado no SAFe foi utilizado a especificação suplementar para definir o requisitos não funcionais. Desta forma, pode-se citar a definição de requisitos legais, reguladores, padrões de aplicação, de sistemas operacionais e ambiente, entre outros.

{\large {\section { Escopo \\ } } }

A fábrica de massas chefNery carece ainda de um site que otimize seu funcionamento. Assim, a equipe de requisitos tem como objetivos definir os requisitos da melhor forma possível e, possivelmente, implementar a aplicação conforme as necessidades e o interesse do cliente. 

{\large {\section { Definições, Acrônimos e Abreviações \\ } } }

SAFe - (scaled agile Framework)

{\large {\section { Usabilidade \\ } } }

Estes requisitos foram desenvolvidos a partir das heurísticas e princípios de usabilidade. 


\begin{table}[H]
                \centering
                \caption{Especificação dos Requisitos de Usabilidade}
                \begin{tabular}{c|p{10cm}}
                    \hline
                    \textbf{Requisito}            & \textbf{Descrição}\\
                    \hline
                    RU01 & A aplicação deverá ser de fácil aprendizado e compreensão por parte do usuário \\ 
                    \hline
                    RU02 & A aplicação deverá ser de fácil aprendizado e compreensão por parte do usuário \\ 
                    RU03 & A aplicação deve fornecer controle e liberdade ao usuário\\ 
                    \hline
                    RU04 & A aplicação deve apresentar consistência e padrões\\
                    \hline
                    RU05 & A aplicação deve prevenir erros por parte do usuário\\
                    \hline
                    RU06 & A aplicação deve apresentar ajuda e documentação de auxílio ao usuário\\
                    \hline                    
                \end{tabular}
            \end{table}


{\large {\section { Confiabilidade \\ } } }

\begin{table}[H]
                \centering
                \caption{Especificação dos Requisitos de Confiabilidade}
                \begin{tabular}{c|p{10cm}}
                    \hline
                    \textbf{Requisito}            & \textbf{Descrição}\\
                    \hline
                    RC01 & A aplicação deve garantir a privacidade dos dados dos clientes cadastrados\\
                    \hline
                    RC02 & A ocorrência de falhas deve ser a menor possível\\ 
                    \hline
                    RC03 & A aplicação deve possibilitar a recuperação dos dados\\
                    \hline
                    RC04 & A aplicação deve apresentar funcionamento estável\\
                    \hline                   
                \end{tabular}
            \end{table}


{\large {\section { Restrições do Design \\ } } }

\begin{table}[H]
                \centering
                \caption{Especificação dos Requisitos de Design}
                \begin{tabular}{c|p{10cm}}
                    \hline
                    \textbf{Requisito} & \textbf{Descrição}\\
                    \hline
                    RD01 & A Aplicação deverá fazer uso de Bootstrap\\ 
                    \hline
                    RD02 & A aplicação deverá utilizar recursos responsivos do bootstrap\\ 
                    \hline
                    RD03 & O repositório oficial da aplicação será o Github\\
                    \hline
                    RD04 & A arquitetura utilizada deve ser o MVC (model - view - controller)\\
                    \hline
                    RD05 &  A aplicação deverá ser feita a  partir da tecnologia ruby on rails\\
                    \hline                    
                \end{tabular}
            \end{table}

{\large {\section { Restrições do Sistema de Ajuda e de Documentação de Usuário on-line \\ } } }

A aplicação deverá apresentar documentação documentação de fácil visualização e disponibilidade. Esta poderá ser disponibilizada em uma página disponível dentro do site, fornecer um canal de comunicação e  até apresentar vídeos de manuseio do produto.\\

{\large {\section { Componentes adquiridos \\ } } }

A aplicação web necessita de um servidor e um domínio para sua hospedagem.\\

{\large {\section { Interfaces \\ } } }

{\subsection {Interfaces com o Usuário\\ }}

A interface do usuário deverá ser disponível para interação por meio de algum navegador web. Além disso, esta interface deverá ser responsiva.\\

{\subsection {Interfaces de Comunicação\\ }}

Para realizar a comunicação entre a aplicação e o dispositivos eletrônico será necessário o uso da internet.\\

{\large {\section { Observações Legais, copyright e Outros  \\ } } }

Esta aplicação se responsabiliza apenas para os critérios que foram definidos neste documento. Portanto, não é assegurado outros tópicos que não estão assegurados neste documento. \\

{\large {\section { Referências \\ } } }

Template Especificação Suplementar. UFPR. Disponível em: \href{http://www.funpar.ufpr.br:8080/rup/webtmpl/templates/req/rup_sspec.htm}{link}. Acesso em: 02 Novembro de 2016.\\

\end{apendicesenv}
