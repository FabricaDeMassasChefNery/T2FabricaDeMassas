\begin{apendicesenv}

\partapendices

\chapter{Documento de Visão}

{\large {\section { Introdução \\ } } }

{\subsection {Finalidade\\ }}
\tab Este documento contém os fatores mais relevantes que levarão à construção do site Chefe Nery. Esse projeto será desenvolvido pelos estudantes das disciplinas requisitos de software que é ministrada pela professora Elaine Venson da Universidade de Brasília, campus Gama/DF (FGA). Serão apresentadas aqui todas as principais características, finalidades e motivações para o desenvolvimento dessa aplicação, assim como as limitações e empecilhos que existem já na concepção da ideia.
\\

{\subsection {O problema\\ }}
\textbf{Discussão:} A fábrica de massas do Chefe Nery é uma micro-empresa que fornece produtos alimentícios em setores como culinária italiana, culinária oriental, entre outros. Realiza pedidos principalmente na Câmara dos Deputados, local de trabalho do dono da empresa e outras regiões próximas ao Plano Piloto e Park Way. Os pedidos são feitos majoritariamente pela plataforma mobile Whatsapp e com registros de pedidos não muito eficiente. \\
\textbf{Formulação do Problema:} 


\newcommand{\nextitem}{\par\hspace*{\labelsep}\textbullet\hspace*{\labelsep}}

\begin{tabular}{|l|p{3in}|}
  \hline
  \textbf{O problema:} & \nextitem Pedidos feitos manualmente
    \nextitem Falta de divulgação da empresa 
    \\ \hline
  \textbf{Afeta:} & \nextitem A gestão de pedidos 
    \nextitem Estoque da empresa 
    \\ \hline
  \textbf{Cujo impacto é:} & \nextitem Possível perda de pedidos
    \\ \hline
  \textbf{Uma boa solução seria:} & \nextitem Um sistema de plataforma web que realize os pedidos de forma autônoma e informasse à empresa sobre as entregas a serem feitas, além de informar aos clientes sobre os produtos preparados pelo chefe Nery.
    \\ \hline

\end{tabular}

{\large {\section { Abordagem do problema \\ } } }

\textbf{Discussão:} A fábrica de massas do Chefe Nery é uma micro-empresa que fornece produtos alimentícios em setores como culinária italiana, culinária oriental, entre outros. Realiza pedidos principalmente na Câmara dos Deputados, local de trabalho do dono da empresa e outras regiões próximas ao Plano Piloto e Park Way. Os pedidos são feitos majoritariamente pela plataforma mobile Whatsapp e com registros de pedidos não muito eficiente. \\

\textbf{Sentença de Posição do Produto:}  \\

\begin{tabular}{|l|p{3in}|}
  \hline
  \textbf{Para} & \nextitem Clientes que passam agora a pedir pelo sistema.
    \nextitem Administradores do estoque, uma vez o próprio sistema permite entregas caso haja estoque. 
    \\ \hline
  \textbf{Que} & \nextitem Precisam realizar pedidos via internet 
    \\ \hline
  \textbf{O produto} & \nextitem A aplicação do Chefe Nery
    \\ \hline
  \textbf{Faz} & \nextitem Expõe os produtos
  \nextitem Gerencia os pedidos
  \nextitem Cadastro de clientes
  \nextitem Realiza pagamentos (caso solicitado)
    \\ \hline

\end{tabular}

{\large {\section { Partes Envolvidas \\ } } }
\textbf{Resumo dos Stakeholders:}  \\

\begin{tabular}{|p{2in}|p{2in}|p{2in}|}
  \hline
  \textbf{Nome} & \textbf{Descrição} & \textbf{Responsabilidades} \\ \hline
  Equipe de Desenvolvimento & Fazem parte do time de desenvolvedores & Responsáveis pela elaboração do programa \\ \hline
  Administradores & Quem administra o estoque e/ou pedidos & Responsável pela gerência de pedidos e estoques da empresa \\ \hline

\end{tabular}

\textbf{Resumo dos Usuários:}  \\

\begin{tabular}{|l|p{3in}|}
  \hline
  \textbf{Nome} & \textbf{Descrição} \\ \hline
  Clientes da fábrica de massas & Os clientes que irão fazer os pedidos pelo site \\ \hline
  Administradores & Os administradores da plataforma \\ \hline

\end{tabular}

\textbf{Principais necessidades dos Usuários:} Os usuários necessitam de um canal que os possibilitem pesquisar os produtos existentes no site, escolher os produtos, assim como sua quantidade, o local de entrega dos produtos e identificar a forma de pagamento que julgarem necessária.   \\

\chapter{Segundo Apêndice}

Texto do segundo apêndice.

\end{apendicesenv}
