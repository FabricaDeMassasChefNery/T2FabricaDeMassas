\begin{apendicesenv}

\partapendices

\chapter{Documento de Visão}

{\large {\section { Introdução \\ } } }

{\subsection {Finalidade\\ }}
\tab Este documento contém os fatores mais relevantes que levarão à construção do site Chefe Nery. Esse projeto será desenvolvido pelos estudantes das disciplinas requisitos de software que é ministrada pela professora Elaine Venson da Universidade de Brasília, campus Gama/DF (FGA). Serão apresentadas aqui todas as principais características, finalidades e motivações para o desenvolvimento dessa aplicação, assim como as limitações e empecilhos que existem já na concepção da ideia.
\\

{\subsection {O problema\\ }}
\textbf{Discussão:} A fábrica de massas do Chefe Nery é uma micro-empresa que fornece produtos alimentícios em setores como culinária italiana, culinária oriental, entre outros. Realiza pedidos principalmente na Câmara dos Deputados, local de trabalho do dono da empresa e outras regiões próximas ao Plano Piloto e Park Way. Os pedidos são feitos majoritariamente pela plataforma mobile Whatsapp e com registros de pedidos não muito eficiente. \\
\textbf{Formulação do Problema:} 

\newcommand{\nextitem}{\par\hspace*{\labelsep}\textbullet\hspace*{\labelsep}}
\begin{tabular}{lp{3in}}
  \textbf{O problema:} & \nextitem Pedidos feitos manualmente
    \nextitem Falta de divulgação da empresa 
    \\ \hline
  \textbf{Afeta:} & \nextitem A gestão de pedidos 
    \nextitem Estoque da empresa 
    \\ \hline
  \textbf{Cujo impacto é:} & \nextitem Possível perda de pedidos
    \\ \hline
  \textbf{Uma boa solução seria:} & \nextitem Um sistema de plataforma web que realize os pedidos de forma autônoma e informasse à empresa sobre as entregas a serem feitas, além de informar aos clientes sobre os produtos preparados pelo chefe Nery.
    \\ \hline

\end{tabular}

\chapter{Segundo Apêndice}

Texto do segundo apêndice.

\end{apendicesenv}
