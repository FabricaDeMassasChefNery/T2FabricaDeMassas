\begin{apendicesenv}

\partapendices

\chapter{Documento de Visão}

\tab Histórico de Revisões


%\begin{tabular}{|l|p{4in}|}
%  \hline
%  \textbf{Revisao} & \nextitem Data \nextitem Descricao & \nextitem Autor
%    \\ \hline
%  \textbf{0.1} & \nextitem 17/10/2016 & \nextitem Versão Inicial & \nextitem Daniel, Miguel Nery
%    \\ \hline
%  \textbf{0.2} & \nextitem 06/11/2016 & \nextitem Problema e Abordagem do Problema & \nextitem Daniel, Miguel Nery
%    \\ \hline
%  \textbf{0.3} & \nextitem 14/11/2016 & \nextitem Abordagem dos Requisitos & \nextitem Daniel, Eduardo
%    \\ \hline
%\end{tabular}


{\large {\section { Introdução \\ } } }

{\subsection {Finalidade\\ }}
\tab Este documento contém os fatores mais relevantes que levarão à construção do site Chefe Nery. Esse projeto será desenvolvido pelos estudantes das disciplinas requisitos de software que é ministrada pela professora Elaine Venson da Universidade de Brasília, campus Gama/DF (FGA). Serão apresentadas aqui todas as principais características, finalidades e motivações para o desenvolvimento dessa aplicação, assim como as limitações e empecilhos que existem já na concepção da ideia.
\\

{\subsection {O problema\\ }}
\textbf{Discussão:} A fábrica de massas do Chefe Nery é uma micro-empresa que fornece produtos alimentícios em setores como culinária italiana, culinária oriental, entre outros. Realiza pedidos principalmente na Câmara dos Deputados, local de trabalho do dono da empresa e outras regiões próximas ao Plano Piloto e Park Way. Os pedidos são feitos majoritariamente pela plataforma mobile Whatsapp e com registros de pedidos não muito eficiente. \\
\textbf{Formulação do Problema:}


\newcommand{\nextitem}{\par\hspace*{\labelsep}\textbullet\hspace*{\labelsep}}

\begin{tabular}{|l|p{3in}|}
  \hline
  \textbf{O problema:} & \nextitem Pedidos feitos manualmente
    \nextitem Falta de divulgação da empresa
    \\ \hline
  \textbf{Afeta:} & \nextitem A gestão de pedidos
    \nextitem Estoque da empresa
    \\ \hline
  \textbf{Cujo impacto é:} & \nextitem Possível perda de pedidos
    \\ \hline
  \textbf{Uma boa solução seria:} & \nextitem Um sistema de plataforma web que realize os pedidos de forma autônoma e informasse à empresa sobre as entregas a serem feitas, além de informar aos clientes sobre os produtos preparados pelo chefe Nery.
    \\ \hline

\end{tabular}

{\large {\section { Abordagem do problema \\ } } }

\textbf{Discussão:} A fábrica de massas do Chefe Nery é uma micro-empresa que fornece produtos alimentícios em setores como culinária italiana, culinária oriental, entre outros. Realiza pedidos principalmente na Câmara dos Deputados, local de trabalho do dono da empresa e outras regiões próximas ao Plano Piloto e Park Way. Os pedidos são feitos majoritariamente pela plataforma mobile Whatsapp e com registros de pedidos não muito eficiente. \\

\textbf{Sentença de Posição do Produto:}  \\

\begin{tabular}{|l|p{3in}|}
  \hline
  \textbf{Para} & \nextitem Clientes que passam agora a pedir pelo sistema.
    \nextitem Administradores do estoque, uma vez o próprio sistema permite entregas caso haja estoque.
    \\ \hline
  \textbf{Que} & \nextitem Precisam realizar pedidos via internet
    \\ \hline
  \textbf{O produto} & \nextitem A aplicação do Chefe Nery
    \\ \hline
  \textbf{Faz} & \nextitem Expõe os produtos
  \nextitem Gerencia os pedidos
  \nextitem Cadastro de clientes
  \nextitem Realiza pagamentos (caso solicitado)
    \\ \hline

\end{tabular}

{\large {\section { Partes Envolvidas \\ } } }
\textbf{Resumo dos Stakeholders:}  \\

\begin{tabular}{|p{2in}|p{2in}|p{2in}|}
  \hline
  \textbf{Nome} & \textbf{Descrição} & \textbf{Responsabilidades} \\ \hline
  Equipe de Desenvolvimento & Fazem parte do time de desenvolvedores & Responsáveis pela elaboração do programa \\ \hline
  Administradores & Quem administra o estoque e/ou pedidos & Responsável pela gerência de pedidos e estoques da empresa \\ \hline

\end{tabular}

\textbf{Resumo dos Usuários:}  \\

\begin{tabular}{|l|p{3in}|}
  \hline
  \textbf{Nome} & \textbf{Descrição} \\ \hline
  Clientes da fábrica de massas & Os clientes que irão fazer os pedidos pelo site \\ \hline
  Administradores & Os administradores da plataforma \\ \hline

\end{tabular}

\textbf{Principais necessidades dos Usuários:} Os usuários necessitam de um canal que os possibilitem pesquisar os produtos existentes no site, escolher os produtos, assim como sua quantidade, o local de entrega dos produtos e identificar a forma de pagamento que julgarem necessária.   \\

{\large {\section { Perspectivas do Produto\\ } } }

{\large {\section { Principais Capacidades\\ } } }

{\large {\section { Restrições do Projeto\\ } } }

{\large {\section { Atributos dos Requisitos\\ } } }
\textbf{Temas de Investimento} \\
\textbf{Épicos} \\
\textbf{Features} \\
\textbf{Histórias de Usuário}\\


F1H1 - Pagar no ato da entrega \\
\\
\tab Data de criação do requisito: 12/11/2016;\\
\tab Início previsto: Sem previsão;\\
\tab Término Previsto: Sem previsão;\\
\tab Valor para o negócio: Alto;\\
\tab Status do Requisito: Aberto;\\
\tab Esforço: Planning Poker será jogado no início de cada sprint para determinar esforço.\\
\tab Descrição: Eu como cliente desejo efetuar a compra, pegando somente na hora da entrega, a fim de pagar por dinheiro ou por cartão.\\
\\
\tab F1H2 - Pagar via sistema online\\
\\
\tab Data de criação do requisito: 12/11/2016;\\
\tab Início previsto: Sem previsão;\\
\tab Término Previsto: Sem previsão;\\
\tab Valor para o negócio: Alto;\\
\tab Status do Requisito: Aberto;\\
\tab Esforço: Planning Poker será jogado no início de cada sprint para determinar esforço.\\
\tab Descrição: Eu como cliente desejo efetuar a compra e pagar por um sistema online(pagseguro), a fim de que o próprio sistema efetue o pagamento.\\
\\
\tab F1H3 - Efetuar pedido checando o estoque\\
\\
\tab Data de criação do requisito: 12/11/2016;\\
\tab Início previsto: Sem previsão;\\
\tab Término Previsto: Sem previsão;\\
\tab Valor para o negócio: Alto;\\
\tab Status do Requisito: Aberto;\\
\tab Esforço: Planning Poker será jogado no início de cada sprint para determinar esforço.\\
\tab Descrição: Eu como administrador do estoque desejo que o próprio sistema avalie a ausência de produtos, a fim de permitir apenas pedidos com a possibilidade de entrega.\\
\\
\tab F1H4 - Escolher endereço de entrega\\
\\
\tab Data de criação do requisito: 12/11/2016;\\
\tab Início previsto: Sem previsão;\\
\tab Término Previsto: Sem previsão;\\
\tab Valor para o negócio: Alto;\\
\tab Status do Requisito: Aberto;\\
\tab Esforço: Planning Poker será jogado no início de cada sprint para determinar esforço.\\
\tab Descrição: Eu como cliente desejo poder escolher o endereço de entrega a fim de que a entrega pode ser em um endereço cujo queira receber os pedidos (ambiente comercial, em domicílio, etc).\\
\\
\tab F1H5 - Ser alertado quanto a pedido\\
\\
\tab Data de criação do requisito: 12/11/2016;\\
\tab Início previsto: Sem previsão;\\
\tab Término Previsto: Sem previsão;\\
\tab Valor para o negócio: Alto;\\
\tab Status do Requisito: Aberto;\\
\tab Esforço: Planning Poker será jogado no início de cada sprint para determinar esforço.\\
\tab Descrição: Eu como cliente e como vendedor desejo ser alertado quando o pedido foi efetuado para que tenha clareza quanto o status de aceitação do pedido.\\
\\
\tab F1H6 - Ajustar o preço conforme opção de entrega\\
\\
\tab Data de criação do requisito: 12/11/2016;\\
\tab Início previsto: Sem previsão;\\
\tab Término Previsto: Sem previsão;\\
\tab Valor para o negócio: Alto;\\
\tab Status do Requisito: Aberto;\\
\tab Esforço: Planning Poker será jogado no início de cada sprint para determinar esforço.\\
\tab Descrição: Eu como administrador desejo que os preços sejam calculados de acordo com o local de entrega a fim de que o próprio sistema seja responsável pela conta do pagamento.\\
\\
\tab F2H1 - Cadastrar usuário\\
\\
\tab Data de criação do requisito: 12/11/2016;\\
\tab Início previsto: 13/11/2016;\\
\tab Término Previsto: 17/11/2016;\\
\tab Valor para o negócio: Alto;\\
\tab Status do Requisito: Em progresso;\\
\tab Esforço: Planning Poker será jogado no início de cada sprint para determinar esforço.\\
\tab Descrição: Como Comprador, quero poder realizar meu cadastro no sistema para poder efetuar pedidos pela plataforma web.\\
\tab Compradores podem ser de diferentes tipos. Campos de cadastro devem exigir o mínimo que torne possível identificar o comprador (Quem, Onde, Contato).\\
\\
\tab F2H2 - Gerenciar conta\\
\\
\tab Data de criação do requisito: 12/11/2016;\\
\tab Início previsto: Sem previsão;\\
\tab Término Previsto: Sem previsão;\\
\tab Valor para o negócio: Alto;\\
\tab Status do Requisito: Aberto;\\
\tab Esforço: Planning Poker será jogado no início de cada sprint para determinar esforço.\\
\tab Descrição: Como Comprador, quero poder gerenciar minha conta para poder visualizar ou atualizar meus dados ou apagar minha conta.\\
\\
\tab F2H3 Visualizar usuários\\
\\
\tab Data de criação do requisito: 12/11/2016;\\
\tab Início previsto: Sem previsão;\\
\tab Término Previsto: Sem previsão;\\
\tab Valor para o negócio: Alto;\\
\tab Status do Requisito: Aberto;\\
\tab Esforço: Planning Poker será jogado no início de cada sprint para determinar esforço.\\
\tab Descrição: Como Administrador do sistema, preciso visualizar todos os usuários cadastrados para poder facilmente.\\
\\
\tab F3H1 - Manter Produto\\
\\
\tab Data de criação do requisito: 12/11/2016;\\
\tab Início previsto: 15/11/2016;\\
\tab Término Previsto: 17/11/2016;\\
\tab Valor para o negócio: Alto;\\
\tab Status do Requisito: Em progresso;\\
\tab Esforço: Planning Poker será jogado no início de cada sprint para determinar esforço.\\
\tab Descrição: Como vendedor responsável pela Fábrica de Massas eu quero criar, atualizar, ler e deletar produtos para que seja possível ter um controle sobre quais e quantos produtos estão disponíveis no meu site.\\
\tab Será necessário identificar cada produto com as seguintes informações:\\
\tab - Nome do Produto;\\
\tab - Ingredientes;\\
\tab - Categoria (Culinária Italianas, Culinária Oriental, Culinária Árabe);\\
\tab - Preço do Produto;\\
\tab - Quantidade Disponível.\\
\\
\tab F3H2 - Calcular Valor do Produto\\
\\
\tab Data de criação do requisito: 12/11/2016;\\
\tab Início previsto: Sem previsão;\\
\tab Término Previsto: Sem previsão;\\
\tab Valor para o negócio: Alto;\\
\tab Status do Requisito: Aberto;\\
\tab Esforço: Planning Poker será jogado no início de cada sprint para determinar esforço.\\
\tab Descrição: Como vendedor do produto desejo saber quanto custa cada produto baseado nos ingredientes e suas quantidades para que seja possível verificar o valor do produto. Esse custo será chamado de custo operacional.\\
\tab Opcionalmente, gostaria de saber o valor do produto baseado também no custo da gasolina do transporte para a entrega e também o próprio custo de tempo gasto pelo Chef Nery na produção do mesmo.\\
\\
\tab F3H3 - Gerar Relatório de Caixa\\
\\
\tab Data de criação do requisito: 12/11/2016;\\
\tab Início previsto: Sem previsão;\\
\tab Término Previsto: Sem previsão;\\
\tab Valor para o negócio: Alto;\\
\tab Status do Requisito: Aberto;\\
\tab Esforço: Planning Poker será jogado no início de cada sprint para determinar esforço.\\
\tab Descrição: Como administrador da Fábrica de Massas quero saber a partir de um gráfico informações de acordo com o tempo acerca do fluxo de caixa de minha empresa envolvendo produtos vendidos para que seja possível ter um melhor controle da minha empresa. Devem ser disponibilizadas opções de visualização diária, semanal e anual.\\
\tab Opcionalmente com projeções futuras dos fluxos de caixa e a possibilidade de comparar fluxos de caixa.\\
\\
\tab F4H1 - Apresentar informações sobre a empresa e contatos\\
\\
\tab Data de criação do requisito: 12/11/2016;\\
\tab Início previsto: 15/11/2016;\\
\tab Término Previsto: 17/11/2016;\\
\tab Valor para o negócio: Alto;\\
\tab Status do Requisito: Em progresso;\\
\tab Esforço: Planning Poker será jogado no início de cada sprint para determinar esforço.\\
\tab Descrição: Como vendedor responsável pela Fábrica de Massas eu quero disponibilizar informações para contato e atalhos para outros meios comunicação como facebook.\\
\tab Será necessário apresentar um atalho para a página oficial da empresa no Facebook e apresentar informações de contato como email e telefone.\\







\chapter{Segundo Apêndice}

Texto do segundo apêndice.

\end{apendicesenv}
